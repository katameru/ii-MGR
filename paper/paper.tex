% Opcje klasy 'iithesis' opisane sa w komentarzach w pliku klasy. Za ich pomoca
% ustawia sie przede wszystkim jezyk i rodzaj (lic/inz/mgr) pracy, oraz czy na
% drugiej stronie pracy ma byc skladany wzor oswiadczenia o autorskim wykonaniu.
\documentclass[english, mgr]{iithesis}

\usepackage[utf8]{inputenc}
\usepackage{listings}

\lstset{ %
  basicstyle=\footnotesize,
  breakatwhitespace=false,
  breaklines=true,
  captionpos=b,
  frame=single,
  language=scala,
  numbers=left,
  numbersep=5pt,
  showstringspaces=false,
  stepnumber=1
}

\usepackage{caption}

%%%%% DANE DO STRONY TYTUŁOWEJ
% Niezaleznie od jezyka pracy wybranego w opcjach klasy, tytul i streszczenie
% pracy nalezy podac zarowno w jezyku polskim, jak i angielskim.
% Pamietaj o madrym (zgodnym z logicznym rozbiorem zdania oraz estetyka) recznym
% zlamaniu wierszy w temacie pracy, zwlaszcza tego w jezyku pracy. Uzyj do tego
% polecenia \fmlinebreak.
\polishtitle    {Wymagający złamania wierszy\fmlinebreak tytuł pracy w~języku polskim}
\englishtitle   {English title}
\polishabstract {\ldots}
\englishabstract{\ldots}
% w pracach wielu autorow nazwiska mozna oddzielic poleceniem \and
\author         {Marcin Januszkiewicz}
% w przypadku kilku promotorow, lub koniecznosci podania ich afiliacji, linie
% w ponizszym poleceniu mozna zlamac poleceniem \fmlinebreak
\advisor        {dr Małgorzata Biernacka}
%\date          {}                     % Data zlozenia pracy
% Dane do oswiadczenia o autorskim wykonaniu
%\transcriptnum {}                     % Numer indeksu
%\advisorgen    {dr. Jana Kowalskiego} % Nazwisko promotora w dopelniaczu
%%%%%

%%%%% WLASNE DODATKOWE PAKIETY
%
%\usepackage{graphicx,listings,amsmath,amssymb,amsthm,amsfonts,tikz}
%
%%%%% WŁASNE DEFINICJE I POLECENIA
%
%\theoremstyle{definition} \newtheorem{definition}{Definition}[chapter]
%\theoremstyle{remark} \newtheorem{remark}[definition]{Observation}
%\theoremstyle{plain} \newtheorem{theorem}[definition]{Theorem}
%\theoremstyle{plain} \newtheorem{lemma}[definition]{Lemma}
%\renewcommand \qedsymbol {\ensuremath{\square}}
% ...
%%%%%

\begin{document}

%%%%% POCZĄTEK ZASADNICZEGO TEKSTU PRACY

%\chapter{Wprowadzenie}

%\ldots

\chapter{Domain-specific languages}

\section{Introduction to DSLs}

Domain-specific languages (DSLs) are computer languages that are designed to express solutions to problems from a specific domain.
By tailoring their notation and capabilities they offer a large improvement in productivity in their application domain.
It's not easy to formulate a strict definition of what is and isn't a DSL.
It can be useful to define DSLs in opposition to general purpose languages (GPLs): DSLs are most expressive when they are used to solve problems in their own domain, while GPLs maintain the same level of expressiveness across many domains.
Even then classifying some languages is problematic.
Let us consider a spectrum of languages: Bash, PostScript, TeX, CSS, regular expressions, or message filters in email software.
Bash is specialized for creating shell scripts, but is still a fully fledged language on its own.
On the other hand we could argue that message filters are barely a language at all.

DSLs are a powerful element of a developer's toolbox, allowing us to write programs that are simpler and much more expressive.
One the one hand they provide abstractions of concepts that are part of the domain by making them structural elements of the language.
On the other hand, they can abstract from the details of the computation part of the solution.
However, this means that in order to create a good DSL we must both: 
\begin{itemize}
	\item have a very good understanding of the domain, so that we can not only model solutions but also a general framework for solutions
	\item be a good developer so that the abstractions we provide are well designed
\end{itemize} 

The general idea of DSLs is to trade off the ease (or even ability) to solve general problems for the value we can add to solving problems from a specific domain \autocite{Mernik:2005}.
Some examples of added value can be:
\begin{itemize}
	\item syntax and semantics that is easy to learn by non-programmers and resembles notation used by domain experts
	\item syntax that includes idioms common in the domain to make code more expressive
	\item programs can be analyzed and verified so that some errors are caught early
	\item program execution can be optimized using domain knowledge
\end{itemize}

For example, SQL is the DSL of relational databases.
It is designed to express the logic of data manipulation and querying in a relational database, and its notation uses the mathematical model of relational algebras.
Most database engines contain query optimizers which use both data available at query definition and at runtime to help performance.
On the other hand, some computations can be very hard to express, and some dialects aren't even Turing complete.

There are many ways to classify DSLs based on their properties \autocite{Gunther:2011}.
We can study their appearance, origin (relation to implementation language), originality, how it's implemented or what is its purpose.
In this paper we focus mainly on Scala, and how it enables DSL creation.
Since the features that are said to make Scala a good language for DSLs relate to building internal DSLs, we will highlight and analyze the patterns used in building abstractions and notations.

\section{Classification}
One way to classify DSLs is to look at the relationship with the language used to implement them \autocite{Artho:2015, Gunther:2011}.
The basic split is between external and internal (or embedded) DSLs.
An external DSL is one that has syntax wholly independent from the implementation language and is a language in its own right.
An embedded DSL can be thought of as a library of the implementation (or host) language which provides abstractions over the domain-specific knowledge.

\subsection{External DSLs}

External DSLs give us the most flexibility of syntax.
Since we are not constrained by the rules of the host language, the source code can be have a form that is the best way of modeling a solution of a domain problem: a set of symbolic equations, files in a specific hierarchy on disk or a list of turtle movement instructions.
Custom syntax usually means a significant rise in implementation complexity and a drop in usability.

\subsubsection{Parsing custom syntax}

The first issue we face in implementing external DSLs is that we need to parse the source files into an internal representation.
Scala has several options for parsing libraries and parser generators which simplify this step significantly.
However, languages with complex (or poorly designed) grammars will still result in complex parsers, and any input that isn't text-based (e.g. image, binary file) will need to be parsed using custom-written code.

\subsubsection{End-user independence}

The source code of programs written in external DSLs is completely separate from the implementation.
To the end user, this means that they don't need to have access to the implementation source code and build pipeline to write programs in the DSL.
This allows use to, for example, embed a DSL in a commercial application.
Customers can then extend the functionality of the program without having access to confidential implementation data.

\subsubsection{Developer tools}

A usability issue that arises when we use an external DSL is that there are no development tools that can help the programmer.
Unless we implement it ourselves, we have no syntax highlighting or auto-completion in an editor.
We also lose the ability to easily debug code.
If we compile the code into an executable we must manually add debug symbols.
If we use an interpreter written in the host language, we must keep information about the piece of source code we are currently running, which sometimes can be extremely difficult.

\subsection{Internal DSLs}

An internal DSL leverages the host language to build programs.
This solution can be severely limiting in terms of syntax because the DSL grammar needs to conform to the grammar of the host language.
However, some languages (Scala included) have ways to adapt their syntax into multiple forms.
Since we use the host language to build programs, a large benefit is that we don't need to worry about parsing.
We can also use the type system of the host to embed a type system for the DSL.

An interesting property of internal DSLs is that they can be designed to inter-operate with other programs written in the host language which helps facilitate code reuse.
For example, we can have in Scala a DSL that allows us to run computations on large remote datasets.
We can then lift functions from a third-party Scala library into our DSL to run complex algorithms on our data.

Internal DSLs can be further split into two groups: shallow and deep embeddings.

\subsubsection{Shallow embeddings}

A shallow embedding (sometimes called piggybacking \autocite{Mernik:2005}) structures and names constructs in the host language in such a way that makes a program written in the DSL map directly onto a program in the host language with the same semantics.

In this situation every stage of a programs lifetime, from parsing to execution is handled by the host language.
This gives us all the benefits that we would have when writing a program in the host language: development tools, an optimizing compiler/interpreter, and externally verified code generation.

\begin{lstlisting}[caption=Example ScalaTest code, label=code:scalatest_sample]
  "A Stack" should "pop values in last-in-first-out order" in {
    val stack = new Stack[Int]
    stack.push(1)
    stack.push(2)
    stack.pop() should be (2)
    stack.pop() should be (1)
  }
  it should "throw NoSuchElementException if an empty stack is popped" in {
    val emptyStack = new Stack[Int]
    a [NoSuchElementException] should be thrownBy {
      emptyStack.pop()
    } 
  }
\end{lstlisting}

In Listing \ref{code:scalatest_sample} we have a ScalaTest\footnote{\url{http://www.scalatest.org/}} test scenario.
We can see that normal Scala code is interspersed with ScalaTest code.
The lines
\begin{verbatim}
"A Stack" should "pop values in last-in-first-out order" in {
it should "throw NoSuchElementException if an empty stack is popped" in {
\end{verbatim}
define test case names, and the lines
\begin{verbatim}
stack.pop() should be (2)
stack.pop() should be (1)
a [NoSuchElementException] should be thrownBy { emptyStack.pop() }
\end{verbatim}
define assertions about code behavior.
When executed, the Scala code is run and the assertions are checked.
We then receive a descriptive report about which tests failed and which succeeded.


\subsubsection{Deep embeddings}

A deep embedding separates the representation of the DSL from its execution.
We use the host language to create the structure of a program, which will be later compiled or interpreted by another part of the DSL.

\begin{lstlisting}[caption=Sample spark code, label=code:spark_sample]
val rdd = sparkContext.
	fromFile("data.csv").
	filter(...).
	map(...)

println(rdd.sum)
\end{lstlisting}

Seen in Listing \ref{code:spark_sample} is an abbreviated Apache Spark\textsuperscript{TM}\footnote{\url{http://spark.apache.org/}} program.
The variable \texttt{rdd} holds metadata that describe the loading and transforming of some data.
This will be compiled to an execution plan, optimized, and run on multiple computers, completely transparently to the user.

While this form of DSL doesn't make any allowances for handy syntax, it can be much more expressive and powerful than a shallow embedding.
A shallow embedding needs to map cleanly into the host language.
A deep embedding doesn't have that restriction, which opens up the possibility of using more powerful abstractions.

Furthermore, while a deep DSL requires the implementation of a custom compiler/interpreter, it gives us the option to embed domain knowledge into the execution of a program.
With an interpreter, we can use information gathered at runtime to make optimizations that would not be possible with a naive interpreter.
If we compile a program to native code, we can generate code that uses assumptions that would not be available to a generic compiler.
We can also generate code for non-standard targets, such as graphics cards, FPGAs, or living cells \autocite{Pedersen:2009}.

\section{Common patterns}

Based on \autocite{Gunther:2011} we describe patterns relating to the notational aspect of DSLs.
These patterns allow a DSL to better model the domain by folding more knowledge into the structure of the language.
They also allow us to abstract away from the language by removing symbols and expressions that are not part of the DSL.

\begin{itemize}
	\item Layout patterns -- these patterns allow us to manipulate how the source code presents visually.
Code layout can represent a hierarchy or other relationship between domain objects.
	\begin{enumerate}
		\item Block scope -- arrange and group expressions in a way that represent their hierarchy in the domain.
Code listing~\ref{code:scalatest_sample} shows us putting each separate test in its own logical scope.
		\item Method chaining -- make actions on domain objects be a result of sequential application of methods.
This way a composite action can be read fluently.
Listing~\ref{code:scalatest_sample} uses this in making test and assertion definitions read like a sentence in English.
		\item Keyword arguments - allow arguments to methods be passed with explicit specification of the parameter name.
This way the user doesn't need to know the implementation detail of the order of parameters in a function call.
	\end{enumerate}
	\item Expression patterns -- these patterns allow us to separate ourselves from the host language by removing details of its operation.
	\begin{enumerate}
		\item Seamless constructor -- keep the user from instantiating objects since things like memory management semantics shouldn't be a part of the DSL.
		\item Operator expressions -- use custom implementations for operators like \texttt{+}, \texttt{-}, \texttt{*} where appropriate.
	\end{enumerate}
	\item Support patterns -- these patterns allow us to further simplify our language:
	\begin{enumerate}
		\item Clean method calls -- remove unnecessary symbols from method calls
		\item Custom return objects -- instead of returning simple values into the host language, wrap results in a custom object.
This can provide additional functionality and be reused in other parts of the DSL.
	\end{enumerate}
\end{itemize}

In the next chapter we will explore how Scala enables the use of these patterns and discuss some of the pitfalls that these features bring.


\chapter{Case studies}

In this chapter we look at two DSLs: one which was written for the purposes of this thesis, and one which is one of the most popular DSLs for Scala today.
We will explore the techniques used to achieve the goals of the language design.

\section{Case one: stats-collector}

As part of this thesis we wrote an example project \texttt{stats-collector}.
It is a Scala implementation of an internal DSL used to compute statistics from arbitrary datasets.
The implementation actually contains two internal DSLs: one deep and one shallow.
The deep embedding describes data extraction and aggregation and provides a way to compile this description into code that computes the statistics.
The shallow embedding provides a shorthand way of constructing these abstract descriptions and attempts to take on a natural syntax.

This code is heavily based on code used in production at \texttt{Neurosoft~Sp.~z~o.o.}.
We present a slightly simpler and cleaner version, not encumbered with various historical necessities.

\subsection{The problem}

We formulate the `stats collection' problem as follows: we have a dataset that is a sequence of rows.
Each row contains a mapping from column name to an arbitrary string value.
A row might not contain any values.
We want a way to easily compute statistics over this dataset.
For example, given a dataset of vehicles that passed through an intersection, we want to compute the average speed of all vehicles, but also the average speed grouped by the type of the vehicle (e.g. car, truck).
We want our DSL to be extensible so that we can easily add more ways of interpreting and aggregating the data.
We also want it to be simple, so that new statistics (e.g. average speed grouped by vehicle color) can be added without much effort by people not familiar with the system.

\subsection{Language design}

First, we want to show an example use of the complete language.
Listing~\ref{code:stats_sample} shows an example configuration that computes four statistics, two grouped by vehicle type and two total.
The \texttt{speed} statistic takes values from the \texttt{vehicle.speed} column, parses them as floats and computes the mean.
The \texttt{speed.events} statistic doesn't parse the column and just lists how many \texttt{vehicle.speed} values were present.
Listing~\ref{code:stats_use} shows how we can use this configuration on actual data.
The compiler is used to create an entry processor from the configuration.
Then we can simply use this processor on an arbitrary dataset and output some result.

\begin{lstlisting}[caption=Defined statistics, label=code:stats_sample]
package pl.wroc.uni.ii.mjanuszkiewicz

import deep._
import deep.compilation._
import shallow._

object SampleStatsConfiguration extends StatsConfiguration {
  select(
    "vehicle.speed" as nothing summarise count into "speed.events",
    "vehicle.speed" as float summarise mean into "speed")

  select(
    "vehicle.speed" as nothing summarise count into "speed.events",
    "vehicle.speed" as float summarise mean into "speed").
		groupBy("vehicle.type" as string)
}
\end{lstlisting}

\begin{lstlisting}[caption=Using statistics, label=code:stats_use]
val stats = SampleStatsConfiguration
val compiler = new ConfigurationCompiler(stats)
val processor = compiler.compiled

val entries = Seq(
  Map("vehicle.speed" -> "50.0", "vehicle.type" -> "car"),
  Map("vehicle.speed" -> "51.0", "vehicle.type" -> "car"),
  Map("vehicle.speed" -> "52.0", "vehicle.type" -> "car"),
  Map("vehicle.speed" -> "53.0", "vehicle.type" -> "truck"))

entries.foreach(processor.processEntry(_))
println(processor.outputResult)
\end{lstlisting}

\subsection{The Deep DSL}

\subsubsection{Describing statistics}
Now that we know how our language should look like, we can show how the embedded description language is structured.
We can see the basic hierarchy: a single aggregation description needs to describe the source data it is using, the aggregation method it is using, and how to identify it.
\begin{lstlisting}[caption=ColumnAggregateDescription, label=code:cad]
case class ColumnAggregateDescription[ColumnType](
    sourceColumnDescription: SourceColumnDescription[ColumnType],
    aggregateBuilder: AggregateBuilder[ColumnType],
    outputName: String) {
  def fresh: ColumnAggregate[ColumnType] = new ColumnAggregate(sourceColumnDescription, aggregateBuilder.fresh, outputName)
}

class ColumnAggregate[-ColumnType](
    sourceColumnDescription: SourceColumnDescription[ColumnType],
    aggregate: Aggregate[ColumnType],
    outputName: String) {
  def add(entry: Entry) = sourceColumnDescription.parseEntry(entry).foreach { aggregate.add(_) }
  def output: Seq[(String, String)] = aggregate.getOutput
}
\end{lstlisting}
Listing~\ref{code:cad} shows the datatype used.
Since we need to aggregate separately for each groupBy value (e.g. if we group by vehicle.type we aggregate data separately for cars, trucks, etc.) we need a way to create new aggregator instances for each group.

\begin{lstlisting}[caption=SourceColumnDescription, label=code:scd]
case class SourceColumnDescription[+Column](columnName: String, columnParser: ColumnParser[Column]) {
  def parseEntry(entry: Entry): Option[Column] = {
    entry.get(columnName) flatMap columnParser.parse orElse columnParser.default
  }
}

trait ColumnParser[+A] {
  def name: String
  def parse(column: String): Option[A]
  def default: Option[A] = None
}
\end{lstlisting}
Listing~\ref{code:scd} shows how we describe column parsing.
To support a new data type in our DSL we just need to provide a simple implementation of the trait ColumnParser.

\begin{lstlisting}[caption=Aggregate, label=code:agg, float, floatlocation=H]
trait AggregateBuilder[-Elem] {
  type Aggr <: Aggregate[Elem]
  def fresh: Aggr
}

trait Aggregate[-Elem] {
  def add(e: Elem): this.type
  def getOutput: Seq[(String, String)]
}
\end{lstlisting}
Listing~\ref{code:agg} shows the interfaces that need to be implemented for aggregates.
We force them to hold mutable state, and we define aggregate output to be simply a sequence of pairs (statistic name, statistic value) already converted to strings.
These traits are very simple so that we can easily add more aggregate types to our DSL as needed.

\begin{lstlisting}[caption=GroupedDescription, label=code:grouped]
case class GroupedDescription(
	grouping: Seq[SourceColumnDescription[_]],
	descriptionsUnderGrouping: SeqColumnAggregateDescription[_]])
\end{lstlisting}

We now have a full image of how a single column aggregation is described.
To describe grouped aggregations we need to collect the aggregations we want to group and the columns by which we group.
Listing~\ref{code:grouped} shows the datatype used to hold grouped descriptions, with some code omitted for brevity.

\subsubsection{Compiling statistics}
We make compiling the statistics simple by not implementing any optimizations.
The \texttt{EntryProcessor} has mutable internal state that maps a pair (parsed grouping values, output name) into a computed aggregate.
Listing~\ref{code:processing} presents the logic of processing entries.

\begin{lstlisting}[caption=Processing entries, label=code:processing]
for each group
  try to parse the grouping columns
  for each aggregate description
    get stored aggregate from internal state or create new one from description
    add parsed value into aggregate
\end{lstlisting}

\subsection{The Shallow DSL}

\begin{lstlisting}[caption=shallow package object, label=code:shallow, float]
package object shallow {
  //Parsers
  def int = IntParser
  def float = FloatParser
  def string = StringParser
  def nothing = EmptyParser

  //Parser default
  implicit class DefaultParser[A](parser: ColumnParser[A]) {
    def or(defaultValue: A): ColumnParser[A] = { new WithDefault(parser, defaultValue) }
  }

  //Aggregators
  def count[A] = new CounterBuilder[A]
  def count[A](p: A => Boolean) = new PredicateCounterBuilder[A](p)
  def mean = new MeanBuilder

  //Describe source column
  implicit class ColumnNameWrapper(columnName: String) {
    def as[A](columnParser: ColumnParser[A]): SourceColumnDescription[A] = SourceColumnDescription(columnName, columnParser)
  }

  //Describe aggregation
  implicit class SourceDescriptionWrapper[A](description: SourceColumnDescription[A]) {
    class IncompleteDescription(aggregateBuilder: AggregateBuilder[A]) {
      def into(outputName: String) = ColumnAggregateDescription(description, aggregateBuilder, outputName)
    }

    def summarise(aggregateBuilder: AggregateBuilder[A]) = new IncompleteDescription(aggregateBuilder)
  }
}
\end{lstlisting}

The shallow DSL for our language consists of two parts: the \texttt{shallow} package object, and the \texttt{StatsConfiguration} trait.
The former provides the implicit classes and named parser and aggregate implementations, while the latter allows us to group descriptions for a dataset in a single scope, and provides the  \texttt{select} and \texttt{groupBy} methods.

The code in listing~\ref{code:shallow} allows us to write \texttt{"a" as float summarise mean into "b"} by expanding it into 
\begin{verbatim}
ColumnAggregateDescription(
  SourceColumnDescription("a", FloatParser),
  new MeanBuilder,
  "b"
)
\end{verbatim}
using the defined implicit classes and names.
The code in listing~\ref{code:config} allows classes that use the trait StatsConfiguration to use expressions \texttt{select(...).groupBy(...)} in their body.
The \texttt{select} method gathers all column aggregate descriptions and stores them in an internal object the we can additionally use \texttt{groupBy} on.
When the \texttt{configure} method is called we convert these internal objects into grouped descriptions for the compiler.

\begin{lstlisting}[caption=StatsConfiguration, label=code:config]
trait StatsConfiguration {
  class StubbedGroupedDescriptions(descriptions: Seq[ColumnAggregateDescription[_]]) {
    private val groupingBuffer: ArrayBuffer[SourceColumnDescription[String]] = ArrayBuffer()
    private[StatsConfiguration] def finishStub() = GroupedDescription(groupingBuffer.toSeq, descriptions)

    def groupBy(groupings: SourceColumnDescription[String]*): this.type = {
      groupingBuffer ++= groupings
      this
    }
  }

  private val groupedDescriptions: ArrayBuffer[StubbedGroupedDescriptions] = ArrayBuffer()
  protected def select(descriptions: ColumnAggregateDescription[_]*): StubbedGroupedDescriptions = {
    val stubbedDescription = new StubbedGroupedDescriptions(descriptions)
    groupedDescriptions += stubbedDescription
    stubbedDescription
  }

  def configure: Seq[GroupedDescription] = groupedDescriptions.map(_.finishStub())
}
\end{lstlisting}

\subsection{Extensions}

This implementation of a \texttt{stats-collector} DSL is meant to show how we can achieve the core functionality of building abstract descriptions of statistics combined with a clean and safe syntax.
The problem formulated here is slightly simpler than the problem it is based on.
There are ways we can enrich this DSL, from refining the current implementation to adding features:
\begin{itemize}
	\item An optimizing compiler - currently the data is processed in an extremely naive way.
By analyzing the descriptions we can optimize the execution by, for example, filtering the collection to contain only the columns we need, caching the results of column parsing, or allowing the data to be processed in parallel.
  \item Composite columns - sometimes we may want to compute a statistic based on more than one column, for example if we measure speed using two methods, we may want to know the distribution of the difference between one speed and the other.
We can implement this by introducing a way to preprocess the data before computing statistics.
  \item Persistence - the dataset changes over time and we want to have up to date statistics, but we can't store the computed aggregates in memory all the time and don't want to redo the entire computation every time the data updates.
We can implement a way to save the state of computed aggregates to disk so that when the dataset is updated, we only load the saved state and add the new value.
\end{itemize}


\section{ScalaTest}

ScalaTest is a popular testing tool for the Scala ecosystem.
It uses a DSL to allow users to cleanly define test cases inline with Scala code.
It supports a variety of testing styles, and exposes extension points so that users can address any special requirements they may have.
It is an interesting example for a DSL, since the problem domain is very closely coupled with the host language.
This freed the creators of ScalaTest from concerns about separating the DSL syntax from the host language syntax.

In this section we will explore how ScalaTest allows users to structure tests, and how it uses Scala features to achieve clean syntax.

\begin{lstlisting}[caption=Simple test, label=code:test, float, floatplacement=H]
class SetSpec extends FlatSpecLike {

  "An empty Set" should "have size 0" in {
    assert(Set.empty.size == 0)
  }
	
}
\end{lstlisting}

A simple test specification is defined in Listing \ref{code:test}.
The \texttt{FlatSpecLike} trait mixes in all the objects and definitions needed for using the DSL.
The implementation is fairly straightforward: an implicit conversion enriches String to contain the \texttt{should} method.
Through chained functions and more implicit conversions, we add the test case description and body to some internal state.
When the test suite is executed, the test body will run, and a report will be generated.

\begin{lstlisting}[caption=Behavior from implicits, label=code:behavior_implicits, float, floatplacement=H]
trait StringShouldWrapperForVerb {
  def should
    (right: String)
    (implicit svsi: StringVerbStringInvocation):
      ResultOfStringPassedToVerb = ...
      
  def should
    (right: BehaveWord)
    (implicit svbli: StringVerbBehaveLikeInvocation):
      BehaveWord = ...
}
\end{lstlisting}

Listing \ref{code:behavior_implicits} shows an interesting pattern in the trait that allows us to call \texttt{should} on Strings.
We would like to use the same syntax in different testing styles, but different styles mean different behavior.
This is solved by making the actual behavior for the \texttt{should} function come from implicits that are in scope.

\begin{lstlisting}[caption=Macros, label=code:macros, float, floatplacement=H]
trait Assertions {
  def assert(condition: Boolean): Assertion =
    macro AssertionsMacro.assert
}
\end{lstlisting}

The assert function in the simple test is actually a scala macro, as seen in Listing \ref{code:macros}.
The purpose of these macros in ScalaTest is to have helpful messages from pure Scala expressions, since a pure Scala assertion could only say that an assertion was false.
The macro attempts to detect simple expressions \texttt{(a == b)} or \texttt{(a.isEmpty)}, and generates the descriptive messages: \texttt{2 was not 1} or \texttt{List(1, 2) was not empty}.
We can also avoid using assertions and explicitly define our expectations by writing code like \texttt{result should have length 3}.
There are also implicit classes provided that expose this functionality.
Additionally, users can implement custom components so these expressions convey more meaning, for example: \texttt{message should have subJson(expectedJson)}



%%%%% BIBLIOGRAFIA

%\begin{thebibliography}{1}
%\bibitem{example} \ldots
%\end{thebibliography}

\end{document}
