\documentclass[english, mgr]{iithesis}

\usepackage[utf8]{inputenc}
\usepackage{listings}
\usepackage{csquotes}
\usepackage[style=numeric, backend=bibtex, block=ragged]{biblatex}
\usepackage{url}
\usepackage{breakurl}
\usepackage[breaklinks]{hyperref}
\usepackage{caption}
\def\UrlBreaks{\do\/\do-}

\lstset{ %
  basicstyle=\ttfamily\footnotesize,
  breakatwhitespace=false,
  breaklines=true,
  captionpos=b,
  frame=single,
  language=scala,
  numbers=left,
  numbersep=5pt,
  showstringspaces=false,
  stepnumber=1
}

\polishtitle    {Języki dziedzinowe w Scali}
\polishabstract {Języki dziedzinowe to języki wyspecjalizowane dla konkretnej dziedziny.
 W tej pracy omówimy wzorce, które mogą być użyte przy projektowaniu DSLa, oraz wpływ jakie mają na język.
 Opiszemy też aspekty języka Scala, które mogą być użyte przy implementacji tych wzorców.}
\englishtitle   {Domain-specific languages in Scala}
\englishabstract{Domain-specific languages are languages specialized for a specific domain.
 In this paper we explore patterns that a DSL can incorporate into its design, and the ways in which they affect the language.
 We also look at how features of the Scala programming language can be used in the implementation of these patterns.}
\author         {Marcin Januszkiewicz}
\advisor        {dr Małgorzata Biernacka}
%\date          {}                     % Data zlozenia pracy

\bibliography{bibliography}
\begin{document}

\chapter{Introduction}

Domain specific languages are small, focused languages that are designed for use in a specific application domain. Thanks to this focus, they allow developers to quickly and efficiently compose solutions while writing clean and expressive code.

Although domain-specific languages are rarely the focus of a programmers work, many regularly use DSLs in various support roles. Most large project use some form of configurable build tool. Make, Maven, Gradle, Cabal, and sbt all use DSLs to configure their build pipeline. Often when some text data needs to be analyzed we write small shell scripts. Using data obtained from GitHub\footnote{http://github.com/} and processed by the site GitHut\footnote{http://githut.info/}, we can see that out of the top 5 most active languages on GitHub one (CSS) is a DSL. The methodology used means that we don't count supplementary uses of DSLs, like makefiles or helper shell scripts, but only the main language used in a repository.

Domain-specific languages are not a new idea. We can point to a language developed in 1957 used to program machine tools \cite{Ross:1978}. There have been many notable DSLs throughout the years; BNF (1959), Logo (1967), SQL Bourne shell (1977), Emacs Lisp (1989) and HTML (1993). Today, with access to powerful programming languages and libraries, creating languages is vastly simplified. This makes DSLs an important way of creating abstractions. While language design can still be complex, implementing the design is easily achievable for a small team, and can be well worth the effort.

One language that has features that are explicitly designed to be used in implementing DSLs is Scala. Designed in 2004 by Martin Odersky at the Programming Methods Laboratory of \'Ecole Polytechnique F\'ed\'erale de Lausanne, it has the explicit design goal of being a "scalable language" that grows with the needs of its users.
A fusions of functional and object-oriented programming with a static type system, Scala enables programming styles that range from mimicking Java syntax to emulating type level constructs from Haskell.

In this paper we will talk about some of the programming patterns used to implement DSLs, and how some Scala features can be used to enable the implementation of these patterns. We will also point out some issues that can arise when using these features, and present implementations of real DSLs in Scala.

\chapter{Domain-specific languages}

\section{Introduction to DSLs}

Domain-specific languages (DSLs) are computer languages that are designed to express solutions to problems from a specific domain.
By tailoring their notation and capabilities they offer a large improvement in productivity in their application domain.
It's not easy to formulate a strict definition of what is and isn't a DSL.
It can be useful to define DSLs in opposition to general purpose languages (GPLs): DSLs are most expressive when they are used to solve problems in their own domain, while GPLs maintain the same level of expressiveness across many domains.
Even then classifying some languages is problematic.
Let us consider a spectrum of languages: Bash, PostScript, TeX, CSS, regular expressions, or message filters in email software.
Bash is specialized for creating shell scripts, but is still a fully fledged language on its own.
On the other hand we could argue that message filters are barely a language at all.

DSLs are a powerful element of a developer's toolbox, allowing us to write programs that are simpler and much more expressive.
One the one hand they provide abstractions of concepts that are part of the domain by making them structural elements of the language.
On the other hand, they can abstract from the details of the computation part of the solution.
However, this means that in order to create a good DSL we must both: 
\begin{itemize}
	\item have a very good understanding of the domain, so that we can not only model solutions but also a general framework for solutions
	\item be a good developer so that the abstractions we provide are well designed
\end{itemize} 

The general idea of DSLs is to trade off the ease (or even ability) to solve general problems for the value we can add to solving problems from a specific domain \autocite{Mernik:2005}.
Some examples of added value can be:
\begin{itemize}
	\item syntax and semantics that is easy to learn by non-programmers and resembles notation used by domain experts
	\item syntax that includes idioms common in the domain to make code more expressive
	\item programs can be analyzed and verified so that some errors are caught early
	\item program execution can be optimized using domain knowledge
\end{itemize}

For example, SQL is the DSL of relational databases.
It is designed to express the logic of data manipulation and querying in a relational database, and its notation uses the mathematical model of relational algebras.
Most database engines contain query optimizers which use both data available at query definition and at runtime to help performance.
On the other hand, some computations can be very hard to express, and some dialects aren't even Turing complete.

There are many ways to classify DSLs based on their properties \autocite{Gunther:2011}.
We can study their appearance, origin (relation to implementation language), originality, how it's implemented or what is its purpose.
In this paper we focus mainly on Scala, and how it enables DSL creation.
Since the features that are said to make Scala a good language for DSLs relate to building internal DSLs, we will highlight and analyze the patterns used in building abstractions and notations.

\section{Classification}
One way to classify DSLs is to look at the relationship with the language used to implement them \autocite{Artho:2015, Gunther:2011}.
The basic split is between external and internal (or embedded) DSLs.
An external DSL is one that has syntax wholly independent from the implementation language and is a language in its own right.
An embedded DSL can be thought of as a library of the implementation (or host) language which provides abstractions over the domain-specific knowledge.

\subsection{External DSLs}

External DSLs give us the most flexibility of syntax.
Since we are not constrained by the rules of the host language, the source code can be have a form that is the best way of modeling a solution of a domain problem: a set of symbolic equations, files in a specific hierarchy on disk or a list of turtle movement instructions.
Custom syntax usually means a significant rise in implementation complexity and a drop in usability.

\subsubsection{Parsing custom syntax}

The first issue we face in implementing external DSLs is that we need to parse the source files into an internal representation.
Scala has several options for parsing libraries and parser generators which simplify this step significantly.
However, languages with complex (or poorly designed) grammars will still result in complex parsers, and any input that isn't text-based (e.g. image, binary file) will need to be parsed using custom-written code.

\subsubsection{End-user independence}

The source code of programs written in external DSLs is completely separate from the implementation.
To the end user, this means that they don't need to have access to the implementation source code and build pipeline to write programs in the DSL.
This allows use to, for example, embed a DSL in a commercial application.
Customers can then extend the functionality of the program without having access to confidential implementation data.

\subsubsection{Developer tools}

A usability issue that arises when we use an external DSL is that there are no development tools that can help the programmer.
Unless we implement it ourselves, we have no syntax highlighting or auto-completion in an editor.
We also lose the ability to easily debug code.
If we compile the code into an executable we must manually add debug symbols.
If we use an interpreter written in the host language, we must keep information about the piece of source code we are currently running, which sometimes can be extremely difficult.

\subsection{Internal DSLs}

An internal DSL leverages the host language to build programs.
This solution can be severely limiting in terms of syntax because the DSL grammar needs to conform to the grammar of the host language.
However, some languages (Scala included) have ways to adapt their syntax into multiple forms.
Since we use the host language to build programs, a large benefit is that we don't need to worry about parsing.
We can also use the type system of the host to embed a type system for the DSL.

An interesting property of internal DSLs is that they can be designed to inter-operate with other programs written in the host language which helps facilitate code reuse.
For example, we can have in Scala a DSL that allows us to run computations on large remote datasets.
We can then lift functions from a third-party Scala library into our DSL to run complex algorithms on our data.

Internal DSLs can be further split into two groups: shallow and deep embeddings.

\subsubsection{Shallow embeddings}

A shallow embedding (sometimes called piggybacking \autocite{Mernik:2005}) structures and names constructs in the host language in such a way that makes a program written in the DSL map directly onto a program in the host language with the same semantics.

In this situation every stage of a programs lifetime, from parsing to execution is handled by the host language.
This gives us all the benefits that we would have when writing a program in the host language: development tools, an optimizing compiler/interpreter, and externally verified code generation.

\begin{lstlisting}[caption=Example ScalaTest code, label=code:scalatest_sample]
  "A Stack" should "pop values in last-in-first-out order" in {
    val stack = new Stack[Int]
    stack.push(1)
    stack.push(2)
    stack.pop() should be (2)
    stack.pop() should be (1)
  }
  it should "throw NoSuchElementException if an empty stack is popped" in {
    val emptyStack = new Stack[Int]
    a [NoSuchElementException] should be thrownBy {
      emptyStack.pop()
    } 
  }
\end{lstlisting}

In Listing \ref{code:scalatest_sample} we have a ScalaTest\footnote{\url{http://www.scalatest.org/}} test scenario.
We can see that normal Scala code is interspersed with ScalaTest code.
The lines
\begin{verbatim}
"A Stack" should "pop values in last-in-first-out order" in {
it should "throw NoSuchElementException if an empty stack is popped" in {
\end{verbatim}
define test case names, and the lines
\begin{verbatim}
stack.pop() should be (2)
stack.pop() should be (1)
a [NoSuchElementException] should be thrownBy { emptyStack.pop() }
\end{verbatim}
define assertions about code behavior.
When executed, the Scala code is run and the assertions are checked.
We then receive a descriptive report about which tests failed and which succeeded.


\subsubsection{Deep embeddings}

A deep embedding separates the representation of the DSL from its execution.
We use the host language to create the structure of a program, which will be later compiled or interpreted by another part of the DSL.

\begin{lstlisting}[caption=Sample spark code, label=code:spark_sample]
val rdd = sparkContext.
	fromFile("data.csv").
	filter(...).
	map(...)

println(rdd.sum)
\end{lstlisting}

Seen in Listing \ref{code:spark_sample} is an abbreviated Apache Spark\textsuperscript{TM}\footnote{\url{http://spark.apache.org/}} program.
The variable \texttt{rdd} holds metadata that describe the loading and transforming of some data.
This will be compiled to an execution plan, optimized, and run on multiple computers, completely transparently to the user.

While this form of DSL doesn't make any allowances for handy syntax, it can be much more expressive and powerful than a shallow embedding.
A shallow embedding needs to map cleanly into the host language.
A deep embedding doesn't have that restriction, which opens up the possibility of using more powerful abstractions.

Furthermore, while a deep DSL requires the implementation of a custom compiler/interpreter, it gives us the option to embed domain knowledge into the execution of a program.
With an interpreter, we can use information gathered at runtime to make optimizations that would not be possible with a naive interpreter.
If we compile a program to native code, we can generate code that uses assumptions that would not be available to a generic compiler.
We can also generate code for non-standard targets, such as graphics cards, FPGAs, or living cells \autocite{Pedersen:2009}.

\section{Common patterns}

Based on \autocite{Gunther:2011} we describe patterns relating to the notational aspect of DSLs.
These patterns allow a DSL to better model the domain by folding more knowledge into the structure of the language.
They also allow us to abstract away from the language by removing symbols and expressions that are not part of the DSL.

\begin{itemize}
	\item Layout patterns -- these patterns allow us to manipulate how the source code presents visually.
Code layout can represent a hierarchy or other relationship between domain objects.
	\begin{enumerate}
		\item Block scope -- arrange and group expressions in a way that represent their hierarchy in the domain.
Code listing~\ref{code:scalatest_sample} shows us putting each separate test in its own logical scope.
		\item Method chaining -- make actions on domain objects be a result of sequential application of methods.
This way a composite action can be read fluently.
Listing~\ref{code:scalatest_sample} uses this in making test and assertion definitions read like a sentence in English.
		\item Keyword arguments - allow arguments to methods be passed with explicit specification of the parameter name.
This way the user doesn't need to know the implementation detail of the order of parameters in a function call.
	\end{enumerate}
	\item Expression patterns -- these patterns allow us to separate ourselves from the host language by removing details of its operation.
	\begin{enumerate}
		\item Seamless constructor -- keep the user from instantiating objects since things like memory management semantics shouldn't be a part of the DSL.
		\item Operator expressions -- use custom implementations for operators like \texttt{+}, \texttt{-}, \texttt{*} where appropriate.
	\end{enumerate}
	\item Support patterns -- these patterns allow us to further simplify our language:
	\begin{enumerate}
		\item Clean method calls -- remove unnecessary symbols from method calls
		\item Custom return objects -- instead of returning simple values into the host language, wrap results in a custom object.
This can provide additional functionality and be reused in other parts of the DSL.
	\end{enumerate}
\end{itemize}

In the next chapter we will explore how Scala enables the use of these patterns and discuss some of the pitfalls that these features bring.


\chapter{Scala features}
The core philosophy of scala is for it to be a 'scalable language', one which can adapt to the needs of its users \autocite{Odersky:2004}. The creators wanted it to be an equally good choice for both small, simple programs and for large components of complex systems. To this end, Scala contains many different language features, some of which are explicitly in service of implementing DSLs. In this chapter we explore how the various patterns used in DSL implementation are supported in Scala.

\section{Parsing}

Scala has good support for creating parsers. There is a Scala standard parser combinator library \texttt{scala.util.parsing}, third-party parsing libraries like \texttt{FastParse}, and a parser generator utility \texttt{parboiled2} written in scala. Thanks to Java interop we can also use parsing libraries for Java. When writing custom parsers, aspects of the language like pattern matching and first-class functions are extremely helpful in keeping the code concise and readable. While this doesn't make the effort needed to create a parser for inconsequential, it does make Scala an attractive choice for implementing external DSLs.

\section{Execution}

While the interpretation of a program in AST form  (usually) doesn't pose much problem, it is much harder to generate an executable binary file. There exist Scala libraries that allow the generation of JVM classfiles such as \texttt{Cafebabe}, but as of writing we have not found a library that allows for the creation of any binary executable format. This is problematic if we want to execute programs written in the DSL in an environment that does not support the JVM.

\section{Internal modeling}

\subsection{Implicits}

The implicits mechanism in Scala has two main components: implicit values/parameters, and implicit conversions. Although these components have different uses and meaning, they usually serve a common goal: to hide away a detail of the implementation that is necessary, but doesn't have to be exposed to the user at all times.

\subsubsection{Implicit values/parameters}

A function can declare that some of its arguments are {\it implicit}. These arguments can be omitted at the call site, and the compiler will perform an {\it implicit lookup}. This action searches in a number of places for a value of the type declared in the argument list, and applies it if found. Implicit values can also be retrieved using the built-in function \texttt{implicitly[T]}. A common use is in functions that want to separate arguments that are logically part of the computation from arguments that represent some execution context. For example, in a library that implements support for asynchronous computations we would have an \texttt{ExecutorService} that provides access to a configured thread pool. Functions from this library would require an \texttt{ExecutorService} as an argument to executo work on other threads. Marking the \texttt{ExecutorService} as implicit allows us to omit it from the function calls and unclutter our code, as shown in listings \ref{code:without_implicit} and \ref{code:with_implicit}

\begin{lstlisting}[caption=Code without implicits, label=code:without_implicit]
def async[T](code: => T)(ex: ExecutorService) = ...
def scheduleOnce[T](code: => T, delay: Duration)(ex: ExecutorService) = ...

val ex: ExecutorService = ...
val k = async(expensiveOperation1())(ex)
val v = async(expensiveOperation2())(ex)
scheduleOnce(doThisAgain(), 1.hour)(ex)
\end{lstlisting}

\begin{lstlisting}[caption=Code with implicits, label=code:with_implicit]
def async[T](code: => T)(implicit ex: ExecutorService) = ...
def scheduleOnce[T](code: => T, delay: Duration)(implicit ex: ExecutorService) = ...

implicit val ex: ExecutorService = ...
val k = async(expensiveOperation1())
val v = async(expensiveOperation2())
scheduleOnce(doThisAgain(), 1.hour)
\end{lstlisting}

\subsubsection{Implicit conversions}
Unary functions can be marked as {\it implicit} to become implicit conversions. This allows us to use expressions of type \texttt{A} as if they had type \texttt{B}, as long as we have an implicit conversion of type \texttt{A => B} can be found by the implicit lookup mechanism. This includes method calls, so we can call methods of type \texttt{B} on values of type \texttt{A}. Implicit conversions can be useful when we are using a library that requires a datatype that is different from what is used internally. Listing \ref{code:conversions} shows an interface that uses byte arrays, but we use implicit conversions to transparently use String within our program.

\begin{lstlisting}[caption=Implicit conversions, label=code:conversions, float]
def put(key: Array[Byte], value: Array[Byte]) = ...
def get(key: Array[Byte]): Array[Byte] = ...

implicit def bytesToString(bytes: Array[Byte]): String = new String(bytes)
implicit def stringToBytes(string: String): Array[Byte]= string.getBytes

put("k1", "v1")
put("k2", "v2")
val result: String = get("k3")
\end{lstlisting}

Implicit parameters and implicit conversion help us in implementing the 'clean method calls' pattern, since methods and parameters that aren't part of the domain but are necessary in the implementation of the language can be sometimes completely omitted.

\subsubsection{Implicit classes}
An implicit class is a shorthand definition of a class with a one-argument constructor alongside an implicit conversion from the argument to that class. This simplifies the implementation of the 'Enriched Class' pattern, which imitates adding methods to existing classes. Listing \ref{code:implicit_classes} shows a simple extension of the Int class. Now the method \texttt{toHexString} is available on any instance of \texttt{Int} as long as the implicit class can be located.

\begin{lstlisting}[caption=Implicit class, label=code:implicit_classes, float]
implicit class RichInt(val self: Int) {
  def toHexString: String = java.lang.Integer.toHexString(self)
}

val i: Int = 5
println(i.toHexString)
\end{lstlisting}

With implicit classes we can further remove non-domain symbols from the language by making implicit the creation of domain objects from objects in the host language.

\subsubsection{Context bounds}
Context bounds are a way to express more information at the type system level. The expressions \texttt{def f[T: Ordered](x: T) = \ldots} means that for a given type\texttt{T}, there exists an implementation of the trait \texttt{Ordered}, i. e. there exists a value of type Ordered[T]. This is actually just syntactic sugar which compiles to \texttt{def f[T](x: T)(implicit ev: Ordered[T])}. This mechanism is often used to emulate the behavior of Haskell type classes. Listing \ref{code:context_bounds} presents the implementation of the Show type class in Scala, which us slightly more verbose than in Haskell.

\begin{lstlisting}[caption=Context bounds, label=code:context_bounds, float]
trait Show[T] {
  def show(t: T): String
}
def print[T: Show](t: T) = {
  val str = implicitly[Show[T]].show(t)
  println(str)
}

implicit val showableInt: Show[Int] = new Show[Int] { def show(i: Int): String = i.toString }
implicit val showableFloat: Show[Float] = new Show[Float] { def show(f: Float): String = f.toString }
print(1)
print(2f)
print("3") //will not compile since there is no implementation od Show[String]
\end{lstlisting}

There is a similar construct called the view bound. The code \texttt{def f[T <\% B](x: T)} means that the type \texttt{T} can be viewed as as type \texttt{B}, and it desugars to \texttt{def f[T](x: T)(implicit ev: T => B)}. However, this is currently deprecated and will be removed in future versions of Scala, as it can be easily reproduce using context bounds.

\subsubsection{Implicit lookup}

The implicit lookup algorithm checks the foloowing locaions\footnote{http://docs.scala-lang.org/tutorials/FAQ/finding-implicits.html}: 

\begin{enumerate}
	\item First look in current scope
	\begin{itemize}
		\item Implicits defined in current scope (\texttt{implicit val x = ...})
    \item Explicit imports (\texttt{import dsl.byte2String})
    \item wildcard imports (\texttt{import dsl.\_})
	\end{itemize}
	\item Now look at associated types in
	\begin{itemize}
		\item Companion objects of a type
		\item Implicit scope of an argument’s type
		\item Implicit scope of type arguments
		\item Outer objects for nested types
	\end{itemize}
\end{enumerate}

The implicits found in group 1 take precedence over implicits found in group 2. If multiple matching implicits are found, we choose the one with the most specific type (this requires another, slightly more technical computation). Otherwise, if none are found or we cannot disambiguate the implicit, the compilation fails.

\subsubsection{Criticisms}

Criticizing implicits is a common thread in online discussions about Scala \autocite{Plush:2015, lakes:2009, Allan:2013, Hale:2011}. While not actually broken, implicits are a powerful and complex feature that is very prone to abuse.

The Scala collections library is an interesting example of implicit parameter (ab)use. Sometimes calles the "Longest suicide note in history" \autocite{lakes:2009}, it makes uses implicit parameters and the \texttt{CanBuildFrom} type class to provide robust operations on collections. Looking at the \texttt{map} method of \texttt{List[A]} which we expect to have the type signature \texttt{def map[B](f: A => B): List[B]}. The actual signature is \texttt{def map[B, That](f: A => B)(implicit bf: CanBuildFrom[Repr, B, That]): That}. Listing~\ref{code:bitset} (taken from a response to \autocite{lakes:2009}) shows that the implicit parameter in \texttt{map} is used to find the best possible return type for that method. While this is a powerful result, it presents less like functionality and more like a leaky abstraction. This is underlined by the fact that in newer versions of Scala the documentation doesn't show the real type, but the one that we expected in the beginning.

\begin{lstlisting}[caption=Bitset example, label=code:bitset]
scala> val bits = BitSet(1, 2, 3)
bits: scala.collection.immutable.BitSet = BitSet(1, 2, 3)

scala> val shifted = bits map { _ + 1 }
shifted: scala.collection.immutable.BitSet = BitSet(2, 3, 4)

scala> val displayed = bits map { _.toString + "!" }
displayed: scala.collection.immutable.Set[java.lang.String] = Set(1!, 2!, 3!)
\end{lstlisting}

Leaky abstractions expose the main problem of implicits, which is the large amount of mental effort needed when reasoning about them. When we encounter a problem or confusion with implicits, we cannot immediately say where a value came from. We need to know the implicit lookup algorithm which, while not complex, is not intuitive. Poorly designed implicit conversions can also hide errors. The implicit lookup algorithm fires every time there is a type error in a program, and tries to find a conversion which will let it compile. We now run a risk of accidentally compiling code that should have caused a type error because it just happened to find a fitting conversion.

\section{Method invocation}

Basic Scala method invocation is identical to Java method invocation. However, there are various alternatives to that notation that Scala allows.

\begin{itemize}
	\item When invoking methods that don't take arguments, we can omit the empty parentheses (\texttt{list.size} instead of \texttt{list.size()}).
	\item When invoking a methods that takes one parameter, we can use the infix notation, that is omit the dot and parentheses ((\texttt{list append 1} instead of \texttt{list.append(1)}))
	\item We can chain infix method invocations. \texttt{a m1 a1 m2 a2 m3} compiles as \texttt{a.m1(a1).m2(a2).m3()}
	\item If we keep parentheses around methods with multiple arguments, we can omit all dots in an expression. \texttt{a m1(a1, a2) m2 a2} will compile to \texttt{a.m1(a1, a2).m2(a2)}
\end{itemize}

These syntax options are crucial in enabling the Method Chaining and Clean Method Calls patterns. However care must be taken to understand the rules of how expressions like these are parsed. Code listings \ref{code:dotless1} and \ref{code:dotless2} show two expressions that look extremely similar. However, the expression in listing \ref{code:dotless1} compiles to \texttt{a.m(m).b}, whereas the other expression is actually two expressions: \texttt{a.m(); m.b()}. In a well designed language this would usually cause a compilation error, but some DSLs might want to reuse the same name in different contexts, which could lead to both code snippets being valid.

\begin{minipage}[t]{0.4\textwidth}
\begin{lstlisting}[label=code:dotless1]
a m
m b
\end{lstlisting}
\end{minipage}\hfill
\begin{minipage}[t]{0.4\textwidth}
\begin{lstlisting}[label=code:dotless2]
a m

m b
\end{lstlisting}
\end{minipage}

\section{Operator overloading}

Scala allows us to define methods whose names contain symbols. When combined with infix method invocation we get a notation that resembles operator overloading, but is in fact just arbitrary method invocation. There are two special cases: 
\begin{itemize}
	\item expressions like \texttt{!b} are translated into \texttt{b.unary\_!()}. This is true only for symbols \texttt{+},\texttt{-},\texttt{\~}, \texttt{!}.
	\item method ends with a colon (\texttt{:}), it binds to the right. This means that \texttt{1 \&: x} is translated into \texttt{x.\&:(1)}
\end{itemize}

There is a small inconsistency between normal methods and symbolic operators. The parser assigns higher priority to operators, so the code \texttt{m get k ! s} parses as \texttt{m.get(k.!(s))}, while \texttt{m get k tell s} will parse as \texttt{m.get(k).tell(s)}. This could be used to avoid some parentheses, but in general can probably cause confusion and bugs.

The ability to define custom operations allows us to implement the Operator Expressions pattern. We can emulate notation used by domain experts and create our own shorthands. The problem with using custom operators is that if we introduce new symbols that are unrelated to the domain we run the risk of creating completely unreadable code. When we're working in the domain of matrix manipulation, we can be reasonably confident as to what effect symbols like \texttt{+} or \texttt{*} have. But in a DSL used to create API endpoints, operators like \texttt{:++>>} or \texttt{<|*|>} are completely opaque, and worst of all, ungoogleable. Martin Odersky recently wrote: "But in retrospect I think maybe they did give a bad example for others to go overboard with symbolic operators." \autocite{Odersky:2016}


\section{Macros}

Macros in Scala are a type-safe way to facilitate compile-time metaprogramming. Unlike text-based substitution seen in languages like C, Scala macros transform the abstract syntax tree of parsed and correctly typed Scala code. The macro implementation is plain Scala code, but since the implementation needs to be available to the compiler, we need to compile it earlier.

We can use macros as a way to incorporate information that is available at compile time into code. This is best shown by \texttt{sqltyped}, a library which infers type information from SQL schema and query statements. Since it does this before compilation, we can have code that has the SQL snippet \texttt{select name, age from person where age > ?} which is automatically transformed into a function of type \texttt{Int => List[\{ age: Int, name: String\}]}.

The macro mechanism in Scala is extremely powerful. It can be used in the implementation of many notation patterns, especially since we can write code that better models the domain, and later transform it with macros to conform better to the capabilities of the host language.

Macro support in Scala is currently marked as experimental and is very actively being worked on. As such, between Scala versions 2.10 and 2.11 there have been several breaking changes and deprecations. Scala 2.12 will not add new features. Instead, work will be focused on a net metaprogramming API, \texttt{scala.meta}. This makes macros seem not to be a safe avenue of feature implementation.

\section{Object initialization expressions}

Scala supports using anonymous classes in programs. An anonymous class is a class that is defined and instantiated inline with code, often because it does not need to be reused. Anonymous classes can still use the normal inheritance mechanisms, so we can have on-off classes that are concise and powerful. In the context of DSLs, we can use this capability to implement the Block Scope and Supertype patterns. On listing~\ref{code:anonymous} we show how a DSL might use anonymous classes. The \texttt{Connection} and \texttt{Encryption} traits are part of the DSL implementation and encapsulate domain models relating to properties of connections. This usage gives us:
\begin{itemize}
	\item Type safety - we can restrict usage of unencrypted connections in by making sure that the connection implements the trait \texttt{Encryption}
	\item Supertype - the \texttt{connect} and \texttt{encrypt} methods can be used only within the scope of the new class body. The domain hierarchy is enforced by the language, which removes clutter and prevents mistakes.
	\item Block scope - connection properties may only be defined within the block used to define the new class instance. This guarantees that any and all information regarding a connection definition will be found in one place.
\end{itemize}

Unfortunately, usage of this pattern relies on a feature of Scala that cannot be abstracted from. The object must be constructed using the \texttt{new} operator and obey all rules of Scala scoping and object creation.

\begin{lstlisting}[caption=Anonymous classes, label=code:anonymous]
trait Connection {
  protected def connect = ...
  protected def on = ...
}

trait Encryption {
  protected def encrypt = ...
}

val conn = new Connection with Encryption {
  connect to "example.com"
  on port 223
  encrypt with TLS
}
\end{lstlisting}

\section{Other features}

\subsection{Type system}

The Scala type system provides many features that aid developers in implementing correct code. Local type inference is important to the notation aspect of DSLs since it allows us to omit type annotation in variable declarations, and possibly keep it out of the DSL entirely. Of course to help us build languages that are robust and extensible we can use all the other features of the Scala type system like variance annotations, type bounds, type members, structural typing, higher-kinded types, or existential types.

The issue with the type system is that with this large and continually growing amount of features it is easier to find strange corner cases that are at best counterintuitive and at worst broken. For example, Scala doesn't have full Hindley–Milner type inference. Instead, an expression is fully typed in the place that it appears in, and sometimes needs to be guided. This can lead to a situation like in listing~\ref{code:inference}, where the variable \texttt{b} was inferred to have type \texttt{X[Nothing]}. This makes it problematic to apply common expression extraction, because we could accidentally remove code that guides the inference.

\begin{lstlisting}[caption=Local inference, label=code:inference]
class X[T] {
  def m(x: T) = this
}

val am = new X().m(1)

val b = new X()
val bm = b.m(1)

val c = new X[Int]
val cm = c(1)
\end{lstlisting}

Listing \ref{code:existentials} is a more convoluted, with its use of type bounds and existential types. However, it shows that sometimes the inference needs to be very explicitly guided to compile simple and evidently correct at all. It is especially confusing for code at line \ref{line:why}, where we have to annotate a value with a type that it already has.


\begin{lstlisting}[caption=Existential inference, label=code:existentials, escapechar=|, float, floatplacement=H]
trait InternalBound

class InternalImpl extends InternalBound

trait Container {
  type Internal <: InternalBound
}

class ContainerImpl extends Container {
  type Internal = InternalImpl
}

object Test {
  type ContainerWithInternal = (T, T#Internal) forSome { type T <: Container; }
  type CS = (A, B) forSome { type A <: Container; type B <: A#Internal }

  def construct[T <: Container](i: T, e: T#Internal): ContainerWithInternal = (i, e)

  //Type error
  //def constructCS1[A <: Container](a: A, b: A#Internal): CS = (a, b) |\label{line:why}|
  def constructCS2[A <: Container](a: A, b: A#Internal): CS = (a, b): (A, A#Internal)

  def main(args: Array[String]): Unit = {

    val container = new ContainerImpl
    val internal = new container.Internal()
    val internalWithCast: container.Internal = new container.Internal()

    //Type errors
    //val res1: ContainerWithInternal = (container, internal)
    //val res2: ContainerWithInternal = construct(container, internal)
    val res3: ContainerWithInternal = construct(container, (internal: container.Internal))
    val res4: ContainerWithInternal = construct[ContainerImpl](container, internal)

    val res5: ContainerWithInternal = (container, internalWithCast)
    val res6: ContainerWithInternal = construct(container, internalWithCast)
    val res7: ContainerWithInternal = construct[ContainerImpl](container, internalWithCast)
  }
}
\end{lstlisting}

\subsection{Named parameters \& default arguments}

The named parameters feature in Scala allows us to indicate which value to use as which parameter in a function. We can also define default values for arguments, so that we can skip them completely. With this we can implement the keyword parameters pattern, as seen in listing \ref{code:named}.

\begin{lstlisting}[caption=Named parameters \& default arguments, label=code:named]
def printName(firstName: String, middleName: String = "", lastName: String = "") = {
  println(s"\$firstName \$middleName \$lastName")
}
printName("John")
printName(firstName = "John", lastName = "Doe", middleName = "F")
\end{lstlisting}

\subsection{Objects as modules}

In Scala, every named function must be a method of a class instance. Normally, this would mean that every methods call would either be called on a named class instance, or be made inside a class, where we have a default reference \texttt{this}. However, Scala provides the ability to import into scope members of a class instance with the same notation used for importing class definitions. Listing \ref{code:modules} shows how we can use this to encode an ML-style module system. A system like this allows us to bring into scope definitions of values and functions without having to create objects or refer to them by name.

\begin{lstlisting}[caption=Modules, label=code:modules]
object module {
  def f = ...
  def g() = ...
}

import module._ //imports all members of object Module
f //Calls Module.f()
\end{lstlisting}

\section{Summary}

We have presented several Scala features that we feel are important components in the implementation of internal DSLs. They are all accompanied by critical opinions and caveats. While no one feature can be pointed to as broken, the accumulated cognitive load when interacting with several features at the same time can prove to be overwhelming. This is compounded by the fact that there is no real concept of 'idiomatic Scala' within the community, which leads to fragmentation in coding styles and opinions (from \cite{Hale:2011} "a best practice emerged: ignore the community entirely"). We leave with two facts useful when designing DSLs: that Scala features should be used very carefully, and that language design is hard.


\chapter{Case studies}

In this chapter we look at two DSLs: one which was written for the purposes of this thesis, and one which is one of the most popular DSLs for Scala today.
We will explore the techniques used to achieve the goals of the language design.

\section{Case one: stats-collector}

As part of this thesis we wrote an example project \texttt{stats-collector}.
It is a Scala implementation of an internal DSL used to compute statistics from arbitrary datasets.
The implementation actually contains two internal DSLs: one deep and one shallow.
The deep embedding describes data extraction and aggregation and provides a way to compile this description into code that computes the statistics.
The shallow embedding provides a shorthand way of constructing these abstract descriptions and attempts to take on a natural syntax.

This code is heavily based on code used in production at \texttt{Neurosoft~Sp.~z~o.o.}.
We present a slightly simpler and cleaner version, not encumbered with various historical necessities.

\subsection{The problem}

We formulate the `stats collection' problem as follows: we have a dataset that is a sequence of rows.
Each row contains a mapping from column name to an arbitrary string value.
A row might not contain any values.
We want a way to easily compute statistics over this dataset.
For example, given a dataset of vehicles that passed through an intersection, we want to compute the average speed of all vehicles, but also the average speed grouped by the type of the vehicle (e.g. car, truck).
We want our DSL to be extensible so that we can easily add more ways of interpreting and aggregating the data.
We also want it to be simple, so that new statistics (e.g. average speed grouped by vehicle color) can be added without much effort by people not familiar with the system.

\subsection{Language design}

First, we want to show an example use of the complete language.
Listing~\ref{code:stats_sample} shows an example configuration that computes four statistics, two grouped by vehicle type and two total.
The \texttt{speed} statistic takes values from the \texttt{vehicle.speed} column, parses them as floats and computes the mean.
The \texttt{speed.events} statistic doesn't parse the column and just lists how many \texttt{vehicle.speed} values were present.
Listing~\ref{code:stats_use} shows how we can use this configuration on actual data.
The compiler is used to create an entry processor from the configuration.
Then we can simply use this processor on an arbitrary dataset and output some result.

\begin{lstlisting}[caption=Defined statistics, label=code:stats_sample]
package pl.wroc.uni.ii.mjanuszkiewicz

import deep._
import deep.compilation._
import shallow._

object SampleStatsConfiguration extends StatsConfiguration {
  select(
    "vehicle.speed" as nothing summarise count into "speed.events",
    "vehicle.speed" as float summarise mean into "speed")

  select(
    "vehicle.speed" as nothing summarise count into "speed.events",
    "vehicle.speed" as float summarise mean into "speed").
		groupBy("vehicle.type" as string)
}
\end{lstlisting}

\begin{lstlisting}[caption=Using statistics, label=code:stats_use]
val stats = SampleStatsConfiguration
val compiler = new ConfigurationCompiler(stats)
val processor = compiler.compiled

val entries = Seq(
  Map("vehicle.speed" -> "50.0", "vehicle.type" -> "car"),
  Map("vehicle.speed" -> "51.0", "vehicle.type" -> "car"),
  Map("vehicle.speed" -> "52.0", "vehicle.type" -> "car"),
  Map("vehicle.speed" -> "53.0", "vehicle.type" -> "truck"))

entries.foreach(processor.processEntry(_))
println(processor.outputResult)
\end{lstlisting}

\subsection{The Deep DSL}

\subsubsection{Describing statistics}
Now that we know how our language should look like, we can show how the embedded description language is structured.
We can see the basic hierarchy: a single aggregation description needs to describe the source data it is using, the aggregation method it is using, and how to identify it.
\begin{lstlisting}[caption=ColumnAggregateDescription, label=code:cad]
case class ColumnAggregateDescription[ColumnType](
    sourceColumnDescription: SourceColumnDescription[ColumnType],
    aggregateBuilder: AggregateBuilder[ColumnType],
    outputName: String) {
  def fresh: ColumnAggregate[ColumnType] = new ColumnAggregate(sourceColumnDescription, aggregateBuilder.fresh, outputName)
}

class ColumnAggregate[-ColumnType](
    sourceColumnDescription: SourceColumnDescription[ColumnType],
    aggregate: Aggregate[ColumnType],
    outputName: String) {
  def add(entry: Entry) = sourceColumnDescription.parseEntry(entry).foreach { aggregate.add(_) }
  def output: Seq[(String, String)] = aggregate.getOutput
}
\end{lstlisting}
Listing~\ref{code:cad} shows the datatype used.
Since we need to aggregate separately for each groupBy value (e.g. if we group by vehicle.type we aggregate data separately for cars, trucks, etc.) we need a way to create new aggregator instances for each group.

\begin{lstlisting}[caption=SourceColumnDescription, label=code:scd]
case class SourceColumnDescription[+Column](columnName: String, columnParser: ColumnParser[Column]) {
  def parseEntry(entry: Entry): Option[Column] = {
    entry.get(columnName) flatMap columnParser.parse orElse columnParser.default
  }
}

trait ColumnParser[+A] {
  def name: String
  def parse(column: String): Option[A]
  def default: Option[A] = None
}
\end{lstlisting}
Listing~\ref{code:scd} shows how we describe column parsing.
To support a new data type in our DSL we just need to provide a simple implementation of the trait ColumnParser.

\begin{lstlisting}[caption=Aggregate, label=code:agg, float, floatlocation=H]
trait AggregateBuilder[-Elem] {
  type Aggr <: Aggregate[Elem]
  def fresh: Aggr
}

trait Aggregate[-Elem] {
  def add(e: Elem): this.type
  def getOutput: Seq[(String, String)]
}
\end{lstlisting}
Listing~\ref{code:agg} shows the interfaces that need to be implemented for aggregates.
We force them to hold mutable state, and we define aggregate output to be simply a sequence of pairs (statistic name, statistic value) already converted to strings.
These traits are very simple so that we can easily add more aggregate types to our DSL as needed.

\begin{lstlisting}[caption=GroupedDescription, label=code:grouped]
case class GroupedDescription(
	grouping: Seq[SourceColumnDescription[_]],
	descriptionsUnderGrouping: SeqColumnAggregateDescription[_]])
\end{lstlisting}

We now have a full image of how a single column aggregation is described.
To describe grouped aggregations we need to collect the aggregations we want to group and the columns by which we group.
Listing~\ref{code:grouped} shows the datatype used to hold grouped descriptions, with some code omitted for brevity.

\subsubsection{Compiling statistics}
We make compiling the statistics simple by not implementing any optimizations.
The \texttt{EntryProcessor} has mutable internal state that maps a pair (parsed grouping values, output name) into a computed aggregate.
Listing~\ref{code:processing} presents the logic of processing entries.

\begin{lstlisting}[caption=Processing entries, label=code:processing]
for each group
  try to parse the grouping columns
  for each aggregate description
    get stored aggregate from internal state or create new one from description
    add parsed value into aggregate
\end{lstlisting}

\subsection{The Shallow DSL}

\begin{lstlisting}[caption=shallow package object, label=code:shallow, float]
package object shallow {
  //Parsers
  def int = IntParser
  def float = FloatParser
  def string = StringParser
  def nothing = EmptyParser

  //Parser default
  implicit class DefaultParser[A](parser: ColumnParser[A]) {
    def or(defaultValue: A): ColumnParser[A] = { new WithDefault(parser, defaultValue) }
  }

  //Aggregators
  def count[A] = new CounterBuilder[A]
  def count[A](p: A => Boolean) = new PredicateCounterBuilder[A](p)
  def mean = new MeanBuilder

  //Describe source column
  implicit class ColumnNameWrapper(columnName: String) {
    def as[A](columnParser: ColumnParser[A]): SourceColumnDescription[A] = SourceColumnDescription(columnName, columnParser)
  }

  //Describe aggregation
  implicit class SourceDescriptionWrapper[A](description: SourceColumnDescription[A]) {
    class IncompleteDescription(aggregateBuilder: AggregateBuilder[A]) {
      def into(outputName: String) = ColumnAggregateDescription(description, aggregateBuilder, outputName)
    }

    def summarise(aggregateBuilder: AggregateBuilder[A]) = new IncompleteDescription(aggregateBuilder)
  }
}
\end{lstlisting}

The shallow DSL for our language consists of two parts: the \texttt{shallow} package object, and the \texttt{StatsConfiguration} trait.
The former provides the implicit classes and named parser and aggregate implementations, while the latter allows us to group descriptions for a dataset in a single scope, and provides the  \texttt{select} and \texttt{groupBy} methods.

The code in listing~\ref{code:shallow} allows us to write \texttt{"a" as float summarise mean into "b"} by expanding it into 
\begin{verbatim}
ColumnAggregateDescription(
  SourceColumnDescription("a", FloatParser),
  new MeanBuilder,
  "b"
)
\end{verbatim}
using the defined implicit classes and names.
The code in listing~\ref{code:config} allows classes that use the trait StatsConfiguration to use expressions \texttt{select(...).groupBy(...)} in their body.
The \texttt{select} method gathers all column aggregate descriptions and stores them in an internal object the we can additionally use \texttt{groupBy} on.
When the \texttt{configure} method is called we convert these internal objects into grouped descriptions for the compiler.

\begin{lstlisting}[caption=StatsConfiguration, label=code:config]
trait StatsConfiguration {
  class StubbedGroupedDescriptions(descriptions: Seq[ColumnAggregateDescription[_]]) {
    private val groupingBuffer: ArrayBuffer[SourceColumnDescription[String]] = ArrayBuffer()
    private[StatsConfiguration] def finishStub() = GroupedDescription(groupingBuffer.toSeq, descriptions)

    def groupBy(groupings: SourceColumnDescription[String]*): this.type = {
      groupingBuffer ++= groupings
      this
    }
  }

  private val groupedDescriptions: ArrayBuffer[StubbedGroupedDescriptions] = ArrayBuffer()
  protected def select(descriptions: ColumnAggregateDescription[_]*): StubbedGroupedDescriptions = {
    val stubbedDescription = new StubbedGroupedDescriptions(descriptions)
    groupedDescriptions += stubbedDescription
    stubbedDescription
  }

  def configure: Seq[GroupedDescription] = groupedDescriptions.map(_.finishStub())
}
\end{lstlisting}

\subsection{Extensions}

This implementation of a \texttt{stats-collector} DSL is meant to show how we can achieve the core functionality of building abstract descriptions of statistics combined with a clean and safe syntax.
The problem formulated here is slightly simpler than the problem it is based on.
There are ways we can enrich this DSL, from refining the current implementation to adding features:
\begin{itemize}
	\item An optimizing compiler - currently the data is processed in an extremely naive way.
By analyzing the descriptions we can optimize the execution by, for example, filtering the collection to contain only the columns we need, caching the results of column parsing, or allowing the data to be processed in parallel.
  \item Composite columns - sometimes we may want to compute a statistic based on more than one column, for example if we measure speed using two methods, we may want to know the distribution of the difference between one speed and the other.
We can implement this by introducing a way to preprocess the data before computing statistics.
  \item Persistence - the dataset changes over time and we want to have up to date statistics, but we can't store the computed aggregates in memory all the time and don't want to redo the entire computation every time the data updates.
We can implement a way to save the state of computed aggregates to disk so that when the dataset is updated, we only load the saved state and add the new value.
\end{itemize}


\section{ScalaTest}

ScalaTest is a popular testing tool for the Scala ecosystem.
It uses a DSL to allow users to cleanly define test cases inline with Scala code.
It supports a variety of testing styles, and exposes extension points so that users can address any special requirements they may have.
It is an interesting example for a DSL, since the problem domain is very closely coupled with the host language.
This freed the creators of ScalaTest from concerns about separating the DSL syntax from the host language syntax.

In this section we will explore how ScalaTest allows users to structure tests, and how it uses Scala features to achieve clean syntax.

\begin{lstlisting}[caption=Simple test, label=code:test, float, floatplacement=H]
class SetSpec extends FlatSpecLike {

  "An empty Set" should "have size 0" in {
    assert(Set.empty.size == 0)
  }
	
}
\end{lstlisting}

A simple test specification is defined in Listing \ref{code:test}.
The \texttt{FlatSpecLike} trait mixes in all the objects and definitions needed for using the DSL.
The implementation is fairly straightforward: an implicit conversion enriches String to contain the \texttt{should} method.
Through chained functions and more implicit conversions, we add the test case description and body to some internal state.
When the test suite is executed, the test body will run, and a report will be generated.

\begin{lstlisting}[caption=Behavior from implicits, label=code:behavior_implicits, float, floatplacement=H]
trait StringShouldWrapperForVerb {
  def should
    (right: String)
    (implicit svsi: StringVerbStringInvocation):
      ResultOfStringPassedToVerb = ...
      
  def should
    (right: BehaveWord)
    (implicit svbli: StringVerbBehaveLikeInvocation):
      BehaveWord = ...
}
\end{lstlisting}

Listing \ref{code:behavior_implicits} shows an interesting pattern in the trait that allows us to call \texttt{should} on Strings.
We would like to use the same syntax in different testing styles, but different styles mean different behavior.
This is solved by making the actual behavior for the \texttt{should} function come from implicits that are in scope.

\begin{lstlisting}[caption=Macros, label=code:macros, float, floatplacement=H]
trait Assertions {
  def assert(condition: Boolean): Assertion =
    macro AssertionsMacro.assert
}
\end{lstlisting}

The assert function in the simple test is actually a scala macro, as seen in Listing \ref{code:macros}.
The purpose of these macros in ScalaTest is to have helpful messages from pure Scala expressions, since a pure Scala assertion could only say that an assertion was false.
The macro attempts to detect simple expressions \texttt{(a == b)} or \texttt{(a.isEmpty)}, and generates the descriptive messages: \texttt{2 was not 1} or \texttt{List(1, 2) was not empty}.
We can also avoid using assertions and explicitly define our expectations by writing code like \texttt{result should have length 3}.
There are also implicit classes provided that expose this functionality.
Additionally, users can implement custom components so these expressions convey more meaning, for example: \texttt{message should have subJson(expectedJson)}


\chapter{Conclusions}

We have used Scala with great success to create a non-trivial custom internal \texttt{stats-collector} DSL.
\texttt{stats-collector} presents a very restricted syntax that is very easy to learn and use by non-programmers, and very easy to extend by programmers.
The language that was developed at \texttt{Neurosoft} that we used as inspiration had the same properties.
It was a huge success, as it took the task of adding new statistics from being a major feature request to requiring a minimal localized amount of code change.
This allowed developers to focus on creating an optimizing expression compiler that reduced computation time, memory pressure and bandwidth use.

We have also seen ScalaTest, a complex internal DSL that is widely used by developers around the world.
The test code written in ScalaTest is clean and expressive.
ScalaTest provides several testing styles and provides detailed and readable reports about results of test runs.
It uses some implicit magic in its implementation, but does it in a way that doesn't leak into user code.
It is a good DSL and good piece of software all around.

However, having such good results is far from guaranteed.
The \texttt{stats-collector} DSL presented here is essentially the third version of the DSL.
The clean implementation is a result of slightly simplifying the domain problem and applying lessons learned from the first two versions, and from studying other DSLs.
The DSL that is currently used in production at \texttt{Neurosoft} abuses operator overloading severely (with expressions like \texttt{"speed" ! "lane" !! "total.\#lane\#.speed" }), the type system reasonably (unwarranted use of existential types), and implicits slightly (poorly designed implicit classes).
And while poor design isn't the fault of the language, features that make it easy to make a poor design and allow it to still be functional are.

A critical view of Scala seems to be common within the industry.
With LinkedIn minimizing their dependence on Scala\footnote{\url{https://www.quora.com/Is-LinkedIn-getting-rid-of-Scala}}, Yammer moving away from it\footnote{\url{https://www.infoq.com/news/2011/11/yammer-scala}}, and a Twitter executive stating that he would not have chosen Scala again\footnote{\url{http://techbus.safaribooksonline.com/video/software-engineering-and-development/9781491924563}}, it is clear that there is a fundamental issue with the design philosophy of Scala that make it unsuitable for some workflows.

Our experience, knowledge, and sensibilities all play a role in the way we write and understand code.
Therefore, any verdict we make about whether a language is good or bad can only be a subjective opinion.
In my opinion, Scala is an interesting language that gives a lot of power to the programmer.
Unfortunately, it gives this power by way of features that, on their own, need to be understood on a very fundamental level to be used correctly.
Even worse, when used together the features feel disjointed, like a collection of language experiments and ideas thrown together without care for coherence.
This reflects on Scalas ability to implement DSLs.
It is a good choice for anyone to implement a simple DSL.
It can be a fantastic tool for creating complex DSLs by experienced Scala developers.
But the possible pitfalls make it hard for me to describe Scala as the go-to tool for creating DSLs.


\appendix
\chapter{Code Listings}
\begin{lstlisting}[caption=Existential inference, label=code:existentials, escapechar=|, float, floatplacement=H]
trait InternalBound

class InternalImpl extends InternalBound

trait Container {
  type Internal <: InternalBound
}

class ContainerImpl extends Container {
  type Internal = InternalImpl
}

object Test {
  type ContainerWithInternal = (T, T#Internal) forSome { type T <: Container; }
  type CS = (A, B) forSome { type A <: Container; type B <: A#Internal }

  def construct[T <: Container](i: T, e: T#Internal): ContainerWithInternal = (i, e)

  //Type error
  //def constructCS1[A <: Container](a: A, b: A#Internal): CS = (a, b) |\label{line:why}|
  def constructCS2[A <: Container](a: A, b: A#Internal): CS = (a, b): (A, A#Internal)

  def main(args: Array[String]): Unit = {

    val container = new ContainerImpl
    val internal = new container.Internal()
    val internalWithCast: container.Internal = new container.Internal()

    //Type errors
    //val res1: ContainerWithInternal = (container, internal)
    //val res2: ContainerWithInternal = construct(container, internal)
    val res3: ContainerWithInternal = construct(container, (internal: container.Internal))
    val res4: ContainerWithInternal = construct[ContainerImpl](container, internal)

    val res5: ContainerWithInternal = (container, internalWithCast)
    val res6: ContainerWithInternal = construct(container, internalWithCast)
    val res7: ContainerWithInternal = construct[ContainerImpl](container, internalWithCast)
  }
}
\end{lstlisting}

\begin{lstlisting}[caption=ColumnAggregateDescription, label=code:cad]
case class ColumnAggregateDescription[ColumnType](
    sourceColumnDescription: SourceColumnDescription[ColumnType],
    aggregateBuilder: AggregateBuilder[ColumnType],
    outputName: String) {
  def fresh: ColumnAggregate[ColumnType] = new ColumnAggregate(sourceColumnDescription, aggregateBuilder.fresh, outputName)
}

class ColumnAggregate[-ColumnType](
    sourceColumnDescription: SourceColumnDescription[ColumnType],
    aggregate: Aggregate[ColumnType],
    outputName: String) {
  def add(entry: Entry) = sourceColumnDescription.parseEntry(entry).foreach { aggregate.add(_) }
  def output: Seq[(String, String)] = aggregate.getOutput
}
\end{lstlisting}

\begin{lstlisting}[caption=SourceColumnDescription, label=code:scd, float]
case class SourceColumnDescription[+Column](columnName: String, columnParser: ColumnParser[Column]) {
  def parseEntry(entry: Entry): Option[Column] = {
    entry.get(columnName) flatMap columnParser.parse orElse columnParser.default
  }
}

trait ColumnParser[+A] {
  def name: String
  def parse(column: String): Option[A]
  def default: Option[A] = None
}
\end{lstlisting}

\begin{lstlisting}[caption=Aggregate, label=code:agg, float]
trait AggregateBuilder[-Elem] {
  type Aggr <: Aggregate[Elem]
  def fresh: Aggr
}

trait Aggregate[-Elem] {
  def add(e: Elem): this.type
  def getOutput: Seq[(String, String)]
}
\end{lstlisting}

\begin{lstlisting}[caption=GroupedDescription, label=code:grouped, float]
case class GroupedDescription(
	grouping: Seq[SourceColumnDescription[_]],
	descriptionsUnderGrouping: SeqColumnAggregateDescription[_]])
\end{lstlisting}

\begin{lstlisting}[caption=shallow package object, label=code:shallow, float]
package object shallow {
  //Parsers
  def int = IntParser
  def float = FloatParser
  def string = StringParser
  def nothing = EmptyParser

  //Parser default
  implicit class DefaultParser[A](parser: ColumnParser[A]) {
    def or(defaultValue: A): ColumnParser[A] = { new WithDefault(parser, defaultValue) }
  }

  //Aggregators
  def count[A] = new CounterBuilder[A]
  def count[A](p: A => Boolean) = new PredicateCounterBuilder[A](p)
  def mean = new MeanBuilder

  //Describe source column
  implicit class ColumnNameWrapper(columnName: String) {
    def as[A](columnParser: ColumnParser[A]): SourceColumnDescription[A] = SourceColumnDescription(columnName, columnParser)
  }

  //Describe aggregation
  implicit class SourceDescriptionWrapper[A](description: SourceColumnDescription[A]) {
    class IncompleteDescription(aggregateBuilder: AggregateBuilder[A]) {
      def into(outputName: String) = ColumnAggregateDescription(description, aggregateBuilder, outputName)
    }

    def summarise(aggregateBuilder: AggregateBuilder[A]) = new IncompleteDescription(aggregateBuilder)
  }
}
\end{lstlisting}

\begin{lstlisting}[caption=StatsConfiguration, label=code:config, float]
trait StatsConfiguration {
  class StubbedGroupedDescriptions(descriptions: Seq[ColumnAggregateDescription[_]]) {
    private val groupingBuffer: ArrayBuffer[SourceColumnDescription[String]] = ArrayBuffer()
    private[StatsConfiguration] def finishStub() = GroupedDescription(groupingBuffer.toSeq, descriptions)

    def groupBy(groupings: SourceColumnDescription[String]*): this.type = {
      groupingBuffer ++= groupings
      this
    }
  }

  private val groupedDescriptions: ArrayBuffer[StubbedGroupedDescriptions] = ArrayBuffer()
  protected def select(descriptions: ColumnAggregateDescription[_]*): StubbedGroupedDescriptions = {
    val stubbedDescription = new StubbedGroupedDescriptions(descriptions)
    groupedDescriptions += stubbedDescription
    stubbedDescription
  }

  def configure: Seq[GroupedDescription] = groupedDescriptions.map(_.finishStub())
}
\end{lstlisting}

\printbibliography

\end{document}
