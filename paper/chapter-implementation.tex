\chapter{Implementation}

As part of this thesis I wrote an example project \texttt{DSLSample}. It is a Scala implementation of an internal DSL used to compute statistics from arbitrary datasets. The implementation actually contains two internal DSLs - one deep and one shallow. The deep embedding describes data extraction \& aggregation and provides a way to compile this description into code that computes the statistics. The shallow embedding provides a shorthand way of constructing these descriptions and attempts to take on a vaguely natural syntax.

This code is heavily based on code used in production at \texttt{Neurosoft~Sp.~z~o.o.}. We present a slightly simpler and cleaner version, not encumbered with various historical necessities.

\section{The problem}

We formulate the 'stats collection' problem as follows: we have a dataset that is a sequence of rows. Each row contains a mapping from column name to an arbitrary string value. A row might not contain any values. We want a way to easily compute statistics over this dataset. For example, given a dataset of vehicles that passed through an intersection, we want to compute the average speed of all vehicles, but also the average speed grouped by the type of the vehicle (e.g. car, truck). We want our DSL to be extensible so that we can easily add more ways of interpreting and aggregating the data. We also want it to be simple, so that new statistics (e. g. average speed grouped by vehicle color) can be added without much effort by people not familiar with the system.

\section{Language design}

First, we want to show an example use of the completed language. Listing~\ref{code:stats_sample} shows an example configuration that computes four statistics, two grouped by vehicle type and two total. The \texttt{speed} statistic take values from the \texttt{vehicle.speed} column, parses them as floats and computes the mean. The \texttt{speed.events} statistic doesn't parse the column and just how many \texttt{vehicle.speed} values were present. Listing~\ref{code:stats_use} shows how we can use this configuration. With the configuration we create a compiler and build an entry processor. Then we can simply use this processor on an arbitrary dataset and output some result.

\begin{lstlisting}[caption=Defined statistics, label=code:stats_sample]
package pl.wroc.uni.ii.mjanuszkiewicz

import deep._
import deep.compilation._
import shallow._

object SampleStatsConfiguration extends StatsConfiguration {
  select(
    "vehicle.speed" as nothing summarise count into "speed.events",
    "vehicle.speed" as float summarise mean into "speed")

  select(
    "vehicle.speed" as nothing summarise count into "speed.events",
    "vehicle.speed" as float summarise mean into "speed").
		groupBy("vehicle.type" as string)
}
\end{lstlisting}

\begin{lstlisting}[caption=Using statistics, label=code:stats_use]
val stats = SampleStatsConfiguration
val compiler = new ConfigurationCompiler(stats)
val processor = compiler.compiled

val entries = Seq(
  Map("vehicle.speed" -> "50.0", "vehicle.type" -> "car"),
  Map("vehicle.speed" -> "51.0", "vehicle.type" -> "car"),
  Map("vehicle.speed" -> "52.0", "vehicle.type" -> "car"),
  Map("vehicle.speed" -> "53.0", "vehicle.type" -> "truck"))

entries.foreach(processor.processEntry(_))
println(processor.outputResult)
\end{lstlisting}

\subsection{Deep DSL}

\subsubsection{Describing statistics}
Now that we know how our language should look, we can show how the embedded description language is structured. We can see the basic hierarchy: a single column aggregation description needs to describe the source data it is using, the aggregation method it is using, and how to identify it. Listing~\ref{code:cad} shows the datatype used. Since we need aggregate separately for each groupBy value (e. g. if we group by vehicle.type we aggregate data separately for cars, trucks, etc.) we need to separate the description from the aggregate instance.

\begin{lstlisting}[caption=ColumnAggregateDescription, label=code:cad]
case class ColumnAggregateDescription[ColumnType](
    sourceColumnDescription: SourceColumnDescription[ColumnType],
    aggregateBuilder: AggregateBuilder[ColumnType],
    outputName: String) {
  def fresh: ColumnAggregate[ColumnType] = new ColumnAggregate(sourceColumnDescription, aggregateBuilder.fresh, outputName)
}

class ColumnAggregate[-ColumnType](
    sourceColumnDescription: SourceColumnDescription[ColumnType],
    aggregate: Aggregate[ColumnType],
    outputName: String) {
  def add(entry: Entry) = sourceColumnDescription.parseEntry(entry).foreach { aggregate.add(_) }
  def output: Seq[(String, String)] = aggregate.getOutput
}
\end{lstlisting}

Listing~\ref{code:scd} shows how we describe column parsing. To support a new data type in our DSL we just need to provide a simple implementation of the trait ColumnParser.

\begin{lstlisting}[caption=SourceColumnDescription, label=code:scd]
case class SourceColumnDescription[+Column](columnName: String, columnParser: ColumnParser[Column]) {
  def parseEntry(entry: Entry): Option[Column] = {
    entry.get(columnName) flatMap columnParser.parse orElse columnParser.default
  }
}

trait ColumnParser[+A] {
  def name: String
  def parse(column: String): Option[A]
  def default: Option[A] = None
}
\end{lstlisting}

Listing~\ref{code:agg} shows the interfaces that need to be implemented for aggregates. We force them to hold mutable state, and we define aggregate output to be simply a sequence of pairs (statistic name, statistic value) already converted to strings. These traits are very simple so that we can easily add more aggregate types to our DSL as needed.

\begin{lstlisting}[caption=Aggregate, label=code:agg, float]
trait AggregateBuilder[-Elem] {
  type Aggr <: Aggregate[Elem]
  def fresh: Aggr
}

trait Aggregate[-Elem] {
  def add(e: Elem): this.type
  def getOutput: Seq[(String, String)]
}
\end{lstlisting}

We now have a full image of how a single column aggregation is described. To describe grouped aggregations we simply need to collect the aggregations we want to group and the columns by which we group. Listing~\ref{code:grouped} shows the datatype used to hold grouped descriptions, with some code omitted for brevity.

\begin{lstlisting}[caption=GroupedDescription, label=code:grouped]
case class GroupedDescription(
	grouping: Seq[SourceColumnDescription[_]],
	descriptionsUnderGrouping: SeqColumnAggregateDescription[_]])
\end{lstlisting}

\subsubsection{Compiling statistics}
We make compiling the statistics simple by not implementing any optimizations. The \texttt{EntryProcessor} has mutable internal state that maps a pair (parsed grouping values, output name) into a computed aggregate. Listing~\ref{code:processing} presents the logic of processing entries.

\begin{lstlisting}[caption=Processing entries, label=code:processing,float,floatplacement=H]
for each grouped statistic description
  try to parse the grouping columns
  for each column aggregate description
    get stored aggregate from internal state or create new one from description
    add parsed value into aggregate
\end{lstlisting}

\subsection{Shallow DSL}

The shallow DSL for our language consists of two parts: the \texttt{shallow} package object, and the \texttt{StatsConfiguration} trait. The former provides the implicit classes and named parser and aggregate implementations, while the latter allows us to group the descriptions in a single class using \texttt{select(...).groupBy(...)}.

\begin{lstlisting}[caption=shallow package object, label=code:shallow,float,floatplacement=H]
package object shallow {
  //Parsers
  def int = IntParser
  def float = FloatParser
  def string = StringParser
  def nothing = EmptyParser

  //Parser default
  implicit class DefaultParser[A](parser: ColumnParser[A]) {
    def or(defaultValue: A): ColumnParser[A] = { new WithDefault(parser, defaultValue) }
  }

  //Aggregators
  def count[A] = new CounterBuilder[A]
  def count[A](p: A => Boolean) = new PredicateCounterBuilder[A](p)
  def mean = new MeanBuilder

  //Describe source column
  implicit class ColumnNameWrapper(columnName: String) {
    def as[A](columnParser: ColumnParser[A]): SourceColumnDescription[A] = SourceColumnDescription(columnName, columnParser)
  }

  //Describe aggregation
  implicit class SourceDescriptionWrapper[A](description: SourceColumnDescription[A]) {
    class IncompleteDescription(aggregateBuilder: AggregateBuilder[A]) {
      def into(outputName: String) = ColumnAggregateDescription(description, aggregateBuilder, outputName)
    }

    def summarise(aggregateBuilder: AggregateBuilder[A]) = new IncompleteDescription(aggregateBuilder)
  }
}
\end{lstlisting}

\begin{lstlisting}[caption=StatsConfiguration, label=code:config,float,floatplacement=H]
trait StatsConfiguration {
  class StubbedGroupedDescriptions(descriptions: Seq[ColumnAggregateDescription[_]]) {
    private val groupingBuffer: ArrayBuffer[SourceColumnDescription[String]] = ArrayBuffer()
    private[StatsConfiguration] def finishStub() = GroupedDescription(groupingBuffer.toSeq, descriptions)

    def groupBy(groupings: SourceColumnDescription[String]*): this.type = {
      groupingBuffer ++= groupings
      this
    }
  }

  private val groupedDescriptions: ArrayBuffer[StubbedGroupedDescriptions] = ArrayBuffer()
  protected def select(descriptions: ColumnAggregateDescription[_]*): StubbedGroupedDescriptions = {
    val stubbedDescription = new StubbedGroupedDescriptions(descriptions)
    groupedDescriptions += stubbedDescription
    stubbedDescription
  }

  def configure: Seq[GroupedDescription] = groupedDescriptions.map(_.finishStub())
}
\end{lstlisting}

The code in listing~\ref{code:shallow} allows us to write \texttt{"a" as float summarise mean into "b"} by translating it into \texttt{ColumnAggregateDescription(SourceColumnDescription("a", FloatParser), new MeanBuilder, "b"} using the defined implicit classes and names. The code in listing~\ref{code:config} allows classes that use the trait StatsConfiguration to use expressions \texttt{select(...).groupBy(...)} in their body. The \texttt{select} methods gathers all column aggregate descriptions and stores them in an internal object the we can additionally use \texttt{groupBy} on. When the \texttt{configure} method is called we convert these internal objects into grouped descriptions for the compiler.

\subsection{Extensions}


